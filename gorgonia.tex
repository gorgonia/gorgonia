\documentclass{article}
\pdfinfo{
  /Title (TITLE)
  /Author (Xuanyi Chew)
}
\usepackage{amsmath}
\usepackage{amsfonts}
\usepackage{hyperref}
\hypersetup{
  colorlinks=true,
  linkcolor=magenta,
}
\urlstyle{same}
\begin{document}
\title{Gorgonia v0.10.0}
\author{Xuanyi Chew \\
  \mbox{}
  chewxy@gmail.com
}
\maketitle

\begin{abstract}
  Gorgonia is a deep learning toolkit.
\end{abstract}

\section{Introduction}

\section{Key Concepts}

There are some key concepts to be introduced in this library. We treat a mathematical expression as a function. Functions are made up of operators and operands. An expression is represented by a graph. A graph contains nodes. Nodes may contain values. Evaluation is the process of computing the final value of the expression. Compilation translates a mathematical expression into machine semantics. This leads to more efficient evaluations.

The subsections below introduce the concepts from in an abstract to concrete manner. Subsections \ref{subsection:expr} to \ref{subsection:val} describes the high level objects we wish to manipulate. Subsections \ref{subsection:graph} to \ref{subsection:valrepr} describes how the high level objects are represented in the library. The Subsections including and following \ref{subsection:eval} describe concepts relating to operational matters.

An attempt has been made to carefully delineate the concepts such that each subsection is independent. However, the concepts are very much intertwined in real life.

\subsection{Expression} \label{subsection:expr}

Gorgonia deals with mathematical expressions. A mathematical expression is a sentence in mathematics that may be evaluated. Examples are as follows:

\begin{align*}
  x + 1 &=\\
  x \times y &= \\
  \sigma(w \times x + b) &=\\
  \frac{\sqrt x}{\|x\|} &=
\end{align*}

Observe that these are not equations: the right hand sides are missing. The $=$ at the end indicates that the expressions are to be evaluated.

\subsection{Operator}

Operators are what performs an operation. Operators have an arbitrary arity. Some operators have a backwards differentiation operator associated with them. From the examples of the previous section, the following are operators:

\begin{table}[ht]
  \centering
  \begin{tabular}{|c|c|}
    \hline
    Operator & Arity\\
    \hline
    $+$ & 2 \\
    $\times$ & 2\\
    $\sigma$ & 1\\
    $\sqrt{}$  & 1\\
    $\|\cdot\|$ & 1\\
    $\frac{\cdot}{\cdot}$ & 2\\
    \hline
  \end{tabular}
\end{table}

\subsection{Function}

Functions take exactly one input and produces exactly one output. Some functions have a ``backward'' function.

The distinction between operators and functions may seem overly fussy and petty. However this adds to our vocabulary a very powerful term. We can now say that neural networks are functions.

\subsection{Backpropagation}

Backpropagation is basically differentiation with regards to a certain input. Given an equation:

$$
y = \sigma(Wx + b)
$$

Backpropagation is simply finding $\frac{\partial y}{\partial W}$, $\frac{\partial y}{\partial x}$, and $\frac{\partial y}{\partial b}$.

\subsection{Variables}

Variables are placeholders in an expression. They may be replaced with values when evaluating an expression.

\subsection{Values} \label{subsection:val}

There are two kinds of Values: tensor and scalar values.

\subsection{Graph} \label{subsection:graph}

A graph is what represents an expression.

\subsection{Node}

All leaves are values after evaluation. All roots are the result of a computation.

\subsection{Values Representation}\label{subsection:valrepr}

\subsection{Evaluation} \label{subsection:eval}

\subsection{Compilation}

\subsection{Virtual Machines}

\subsection{Engines}


\end{document}
